\documentclass{article}
\textwidth 7.5in
\textheight 9.5in
\oddsidemargin -.5in
\evensidemargin -.5in
\topmargin-.75in   
\usepackage{graphics, graphicx, color, xcolor}   
\usepackage{amsmath, amssymb, amsthm} \usepackage{algorithm2e}
\usepackage{comment} 
\usepackage{tikz}
\usepackage{listings}
\usepackage{color}
\newenvironment{solution}
  {\begin{proof}[Solution]}
  {\end{proof}}
 \definecolor{dkgreen}{rgb}{0,0.6,0}
\definecolor{gray}{rgb}{0.5,0.5,0.5}
\definecolor{mauve}{rgb}{0.58,0,0.82}

\lstset{frame=tb,
  language=Python,
  aboveskip=3mm,
  belowskip=3mm,
  showstringspaces=false,
  columns=flexible,
  basicstyle={\small\ttfamily},
  numbers=none,
  numberstyle=\tiny\color{gray},
  keywordstyle=\color{blue},
  commentstyle=\color{dkgreen},
  stringstyle=\color{mauve},
  breaklines=true,
  breakatwhitespace=true,
  tabsize=3
}


\begin{document}
\begin{center}
    \large\textbf{Algorithms}\\
    \Large\textbf{Vertex Cover Project Worksheet}
\end{center}

Suppose you have the following graph:
\\
\begin{center}
    

\begin{tikzpicture}[node distance={20mm},main/.style = {draw, circle}] 
\node[main] (1) {$1$};
\node[main] (2) [right of=1] {$2$};
\node[main] (3) [below left of=1]{$3$};
\node[main] (4) [below of=1]{$4$};
\node[main] (5) [below of=2]{$5$};

\draw[-] (1)--(3);
\draw[-] (2)--(4);
\draw[-] (1)--(5);
\draw[-] (5)--(2);

\draw[-] (5)--(4);
\draw[-] (3)--(4);
\draw[-] (1)--(4);
;
\end{tikzpicture} 
\end{center}
\begin{enumerate}
    \item List all possible vertex covers, and the number of vertices for each. Which one(s) are optimal?
    \newline \newline \newline \newline \newline 
    \newline \newline \newline \newline \newline
    \newline \newline \newline \newline \newline
    \item Use the greedy algorithm choosing the vertices with the most uncovered edges, what is the result?
    \newline \newline \newline \newline \newline
    \newline \newline \newline \newline \newline
    \item The greedy algorithm is actually pretty good. In many cases it produces the optimal solution, however this is not always the case. Show that the greedy algorithm does not always return the optimal solution.\\ \textit{Hint:} Find a counterexample.
\end{enumerate}

\end{document}
